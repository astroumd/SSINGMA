% This is the ngVLA_chapt_template.tex file, based on the aspauthor.tex LaTeX file
% Version 1.1: 16 January 2018
% v1.1 changes:  Inclusion of abstract as this will be searchable by ADS.  

% For more instruction on putting together you chapter (e.g., Figures, Tables, Lists, etc.)
% Please see the aspauthor.tex for more information on 
% Copyright 2014, Astronomical Society of the Pacific Conference Series
% Revision:  14 August 2014

% To compile, at the command line positioned at this folder, type:
% latex aspauthor
% latex aspauthor
% dvipdfm aspauthor
% This will create a file called aspauthor.pdf.

%##############################
% INSTRUCTIONS TO THE AUTHOR
%------------------------------------------------
% The Science Book aims to describe in detail the diverse science cases
% for which the ngVLA will be capable of astronomical breakthroughs. Each
% contribution will take an idea and develop it in 3 to 7 pages, including
% at least one illustration. The contributions will describe in detail the
% scientific rationale including:
%
% 1) Description of the problem:
%   What is the goal? (e.g., study cold molecular outflows from galaxies)
%
% 2) Scientific importance:
%   Why is this a problem worth pursuing? (e.g., they play a key role in enriching the inter-galactic medium)
%
% 3) Astronomical impact:
%   What impact will solving the problem have on the broader astronomy questions? (e.g., they are thought to be a key element of feedback and they play a fundamental role at shaping galaxies)
%
% 4) Anticipated results: 
%  What will be some direct quantitative and qualitative results that will be obtained by the observations?   (e.g., they will mass outflow rates, escape fractions, whether they are closer to energy or momentum conserving, etc)
%
% 5) Limitations of current astronomical instrumentation:
%  Why can't the problem be solved with current/planned instruments? (e.g., mass and surface brightness sensitivity are limited)
%
% 6) Connection to unique ngVLA capabilities:
%   Why is the ngVLA the instrument to address this problem? (e.g., obtaining HI, OH, and CO measurements with unique sensitivity)
%
% 7) Experimental layout:  
%  What are the precise measurements required to solve the problem?   (e.g., sampling physical scales of 100 pc to 100 Mpc and imaging hundreds of galaxies)
%
% 8) Complementarity: 
%   How will the ngVLA information complement or be complemented by observations from other facilities?   (e.g., a key goal for ATHENA will be to study the very hot phase, but the cold phases dominate the mass budget)
%
%
% Authors can either:
%  - Adopt the 8 suggested categories as their section titles, or 
%  - Choose their own section titles as long as they cover the 8 suggested categories.
%
% All contributions should be written in LaTeX using the ASP style file.
% A first rough draft of the Science Book chapter is due on December 22nd,
% and should be either submitted to the corresponding Science Working
% Group chair
% <https://science.nrao.edu/futures/ngvla/science-advisory-council>, the
% ngVLA Project Scientist, or uploaded to the ASP website.
% The deadline for the final version of the chapters will be March 31st,
% 2018.
%
%##############################

\documentclass[11pt,twoside]{article}
\usepackage{./asp2014}

\aspSuppressVolSlug
\resetcounters

\bibliographystyle{asp2014}

\markboth{P.J. Teuben, J.A. Turner, and S. Negusssi }{Short Spacing Issues for Extended Emission} 

\begin{document}

% \title{Short Spacing Issues for the Mapping of Milky Way Extended Emission and Nearby Galaxies}
\title{Short Spacing Issues for the Mapping of Extended Emission}
\author{Peter J. Teuben$^1$, Jordan Turner$^2$ and Sara Negussi$^1$
\affil{$^1$Astronomy Department, College Park, MD, USA; \email{teuben@astro.umd.edu}}
\affil{$^2$Department of Physics and Astronomy, Laramie, WY, USA; \email{jturne19@uwyo.edu}}
\affil{$^3$Astronomy Department, College Park, MD, USA; \email{sara.d.negussie@gmail.com}}}

% This section is for ADS Processing.  There must be one line per author.
\paperauthor{Peter J. Teuben}{teuben@astro.umd.edu}{0000-0003-1774-3436}{University of Maryland}{Astronomy Department}{College Park}{MD}{20742}{USA}
\paperauthor{Jordan A. Turner}{jturne19@uwyo.edu}{}{University of Wyoming}{Department of Physics and Astronomy}{Laramie}{WY}{82071}{USA}
\paperauthor{Sara Negussie}{sara.d.negussie@gmail.com}{}{University of Maryland}{Astronomy Department}{College Park}{MD}{20742}{USA}

%Please include a brief abstract that will be used by ADS for searching purposes.  
\begin{abstract}

In this memo we study the fidelity of images obtained with various
combinations of the ngVLA Core, Short Baseline Array and a total power
dish, and make recommendations what configuration is best suited to
recover large scale stuctures in wide field imaging.


\center{{\it \bf This is still a draft, do not distribute.}}


\end{abstract}

\section{Introduction}

The next generation Very Large Array will provide unprecedented
resolution and sensitivity at radio frequencies from 1 to 115 GHz.
For studies of nearby galaxies (z<0.1), the ngVLA as conceived will
enable efficient mapping of neutral hydrogen, carbon monoxide, and the
cm to mm-radio continuum, revealing the broad brush physical
characteristics of molecular clouds and galaxy disks as well as small
structural features such as molecular clouds cores and protoplanetary
disks on milli-arcsecond (parsec) scales.  However, like any
interferometric array, the ngVLA will not cover all possible
baselines, and thus it will be limited in its ability to recover the
true flux for all observed spatial scales.  Full flux recovery is
particularly important for recovering the spatial structure in line
and continuum observations of extended objects, and for measuring
accurate fluxes and precise line ratios for any type of extended
emission. Since these types of observations are at the core of many of
the science use cases the ngVLA is designed to tackle, it is important
to design in the correct flux recovery capabilities. For its baseline
design the ngVLA would be, otherwise, blind to structures larger than
20\" in the 3mm window.


The alternatives for full flux recovery are of two types: 1) a
heterogeneous array with at least two different antenna sizes, where a
fraction of the collecting area is in small diameter dishes that are
closely packed, and a fraction of the large diameter dishes are
equipped for total-power (single-dish) operation, or 2) a homogeneous
array with a large single-dish antenna equipped with an array receiver
for independent fast mapping. Careful analysis of relevant simulations
are needed to best understand the optimal array configuration and
ancillary benefits of either one of these options.


This study has the goal of comparing the relative merits of both
options, and evaluating the best architecture in each case. A ngVLA
study by D. Frayer has produced a memo (ngVLA memo no. 14) looking at
the requirements for a ``large single-dish'' option,
although the evaluation did not look in detail at metrics such as
``image fidelity'' for realistic astronomical
targets. The purpose of the study outlined here is to take where that
memo left to expand the study for the single-dish alternative, while
at the same time producing a full study for the heterogeneous array
alternative and discussing the relative technical and operational
merits of both options.


\section{Study Outline (from proposal)}


\subsection{Performance of Heterogeneous Array with Total Power Antennas}

\begin{enumerate}
  
\item What are the optimal diameters for the small antennas?

\item What is the optimal total collecting area for the small-diameter array?
  
\end{enumerate}

We will assume that the nominal design for the main (large) dishes of
the ngVLA is 18m diameter with a minimum separation of 1.5 times the
antenna diameter, and study the optimal sizes and number for the small
diameter dishes. We will assume that a minimum dish diameter of 5m is
necessary to fit a standard receiver package which should be the same
as used in the large antennas, and a separation of 1.5 times the
diameter at minimum to avoid collisions and collision detection
(although arrays such as BIMA routinely operated within
``collision range,'' the extra collision detection and
avoidance equipment and algorithms are undesirable as they introduce
cost, performance, and maintenance penalties).

\subsection{What is the optimal collecting area of large antennas equipped with total-power capabilities?}


We will look at different metrics for evaluating the optimal
collecting area of small vs. large diameter arrays, including the idea
that the surface brightness sensitivity of the tapered main array
should match that of the small-dishes array and the total power
antennas (which we understand is the algorithm ALMA employs to
determine the relative integration times). We will assume that the
integration times will have to be similar.


\subsection{What is the image fidelity performance?}


In order to assess the image reconstruction in a quantitative way, we
will use a number of realistic models, ranging from a rotating galaxy,
to Koda \& Teuben (2019, in prep) high resolution power spectrum
molecular cloud simulation. Measures such as flux recovery, power
spectrum and rotation curve can be extracted to assign a value to the
fidelity. We will either use the traditional ``feather''
technique of combining the different resolution data, or if
appropriate, the new TP2VIS joint-deconvolution method, but not
compare the two methods.


\subsection{What are the gain stability requirements if any?}


The effects of atmospheric fluctuations can be added for this as well,
although likely they are unimportant for the baselines we would
consider. Gain stability requirements on this assessment will be
determined in the process (cf. Mason 2013), if applicable.


\subsection{What are the likely observing strategies?}


We also want to address observing strategies in the difference
scenarios. For example, if the correlator allows observations with
cross-correlations of the large (18m) and small (6m) dishes, this will
improve the calibration, as well as the net collecting area, but it
would require simultaneous observing.


Similarly, will the TP antennas go in and out of the ngVLA array? Or
stay by themselves (the current ALMA model, although plans are to
integrate them). This impacts collecting area optimization/cost.


Performance of a Homogeneous Array with a Large Single Dish Complement

A large single dish (SD) with array receivers, as was discussed by
Frayer (2017) will also be considered in our simulations. We will
compare the various aspects as we discussed with the heterogenous
array and contrast them. Frayer discussed one particular source, we
would expand this to the larger set of objects we are considering for
this study. As a result an optimal size of the SD , for given number
of array receivers, can be expected from this comparison.  An
advantage of a large SD is that it will be a very nice stand-alone
instrument. On the other hand, it will require a completely different
type of maintenance and software.  All of these will need to be
contrasted as well in the operating budget. The SD part of this study
will include points D-F above (as points A-C are essentially included
in the existing Frayer 2017 work).


\section{Workplan (from proposal)}

Teuben will be responsible for setting up the simulation scripts (in a
github repo) as well as collecting the sample input models for the
simulations. Teuben and Dale's graduate student (Jordan Turner) will
run and analyze a large series of simulation, supervised by Teuben and
Dale. We will have regular bi-weekly telecoms, and plan to meet face
to face twice, once at the start of the project at UW, and once when
Turner visits Teuben at UMD. Results will be presented at the June
2018 meeting by the UW team and a final memo produced by the end of
Summer 2018. We will be in close contact with Brian Mason at NRAO, who
has carried out similar studies for ALMA, and with David Frayer at
GBO, author of the SD memo.  From previous experiences by the PI, a
study such as this one will likely push the boundaries of the CASA
environment, and will need close interactions with the CASA team.




\section{Deliverables  (from proposal)}


We will deliver a memo (this memo) with the points made above, and images to show
the total flux recovery and a measure of their recovered structure. In
addition we plan to publish a set of ``models'' that can be used to
benchmark the performance for future designs. We will also deliver a
github repository with the scripts to set up and run such a suite of
simulations. We will use CASA for the simulation scripts. The github
repo will be open to the public from the start of the project.



\section{Goodness of Simulation}

To compare a model ($M$) and a simulation ($S$), one normally would
just look at the difference map, $D$:

$$
	D_{ij} = M_{ij} - S_{ij}
$$

and could use the RMS, $\sigma_D$, as a measure for the comparison, perhaps
normalized to the model peak.

However, a common measure of the goodness of an image is the
Fidelity. This is defined from the fidelity map:

$$
   F_{ij} = {{|M|} \over {|M-S|} }
$$

instead

$$
   F_{ij} = {    {|M|} \over {max(|D|, \sigma_D/\sqrt{2})} }
$$

which gives a localized dynamic range how well emission is
recovered. It is easy to see that in places where the model and
simulation cross over each other, the fidelity map will have sharp
features of arbitrarely large fidelity.

One commonly computes a scalar fidelity, with the idea of giving one
simple measure how good the simulation approaches the model. One
version is the one uses in CASA's {\tt simobserve}

$$
	F_s = {   {max(|M|} \over {\sigma_D}  }
$$

Another uses the cumulative distibution of fidelities.

And in Pety et al (2004?) the fidelities are binned in 4 pieces, where
the fidelity is considered where the model map is between 0, 0.03,
0.1, 0.3 and 1 of the peak value in the map. For each of these the
median fidelity is computed, of these 4 values the median is computed
again (or is it the mean).

MADM = medabsdevmed = Median of Absolute Deviations from the Median.


\section{Models}

\subsection{Rotating Galaxy with a Flat Rotation Curve}

Sadly we have to use NEMO, because there is no simple way, or we're
lazy to write this in python?


We parametrize the galaxy with a dimensionless number that tells how
well the density distribution matches to the point where the rotation
curve becomes flat. This way we can study extreme cases such as a
constant surface brightness disk vs. an exponentation disk.

Two parameters interesting for a bottom line science question would be
the central velocity gradient ($dV/dR$) and the value and slope of the
outer disk rotation curve, which establishes the amount of dark
matter. The central velocity gradient can also be heavily influenced
by beam smearing. This is a well understood effect, and needs to be
taken into account.

\subsection{ISM hierarchical structure}

Here we study a continuum map of hierarchical structure of the ISM
with a given power law index, where the obvious science number to be
retrieved is that power law index.

We can use Koda's random map generator, but notice that Koda et al
(2019) use two tricks: the main emission is centered (this is a
somewhat subjective procedure), and the emission is given another
transformation to make it look more like the true ISM.  This doesn't
make sense to me yet, could that not be done with another power law
index?

It is certainly true  we don't want sharp edges in maps, since they can give stronger responses by the
interferometer that we don't need to contribute in for example the Fidelity.



\section{Simulations}


We limit ourselves understanding the ngVLA response to large scale
structure, and avoid extreme non-gausssian beams that one gets with
for example the Plains configuration where the antennae are out to
300km (see discussions by Carilli et al. 2019?)  The Core
configuration will limit us to 94 antennas of 18m and about 1km
diameter, but has no baselines short of about 30m.  The Short Baseline
Array (SBA) consists of 19 dishes of 6m each, filling baselines
between 12m and 60m. The SBA and Core thus have some overlap between
30m and 60m



We also want to limit the declination of the source, as for lower
declinations the first effect is to create shorter baselines,
partially the missing short spacing problem go away, but adding in
less ideal beams and shadowing effects.

We fixed the frequency at 115 GHz, at one extreme end of the frequency
range of the intended ngVLA range. For the first set of simulations,
where no artificial noise is added, this is simply a scaling
factor. If we need to incoorporate realistic noise, the dimensionless
noise factor that we can use in the simulations will need to be
calibrated whatever the receivers at a specific frequency



We used the QAC (Quick Array Combination) toolkit (Teuben, 2019). QAC
is a simplified frontend to {\tt simobserve}, {\tt clean} and other
CASA tools and tasks to simplify writing such simulation scripts.  See
also Appendix A for some details.

\subsection{noise}

We follow the {\tt simplenoise} procedure in CASA's {\tt sm} tool.  In
this procedure a random noise value is added to each visibility. Since
at this stage it is not known what the resulting noise in Jy/beam will
be, this simplenoise is first added to a zero amplitude model, the RMS
in the dirtymap will then result in the scaling factor that will
guarentee the requested noise. See also Carili et al. (2017).
The ngVLA noise calculator (in
preparation) can then be used to get the appropriate noise value for a
given observing setup.

In our case of a Core configuration this is probably fairly good for
low frequency observations, but more sophisticated models where
amplitude and phase noise can depend on baseline.

\section{Results}

\subsection{Mapping2}

Sara was using mapping2 to study the fidelity as function of dish size for
only the Core configuration, no SBA. See figure~\ref{fig1}.


\articlefigure{fidelity1.png}{fig1}{Fidelity of Core and a single TP dish}

\subsection{Questions-1}


Questions from the mapping2 study we need to answer:

\begin{itemize}

  \item How does $F$ depend on the model (randomization). Here we can pick one
    (dish,grid) combination and see its variation
  \item How does $F$ depend on the noise. Here we should pick on grid combination
    but make the dish-F plot for a few noise levels.
  \item How ...
  \item How does $F$ depend on the slope of the structure function. This is
    a more interesting question that helps calibrating F with an astronomical
    product (k)
  
\end{itemize}

\subsection{Questions-2}

Back to the vla2 script, we need to address:

\begin{itemize}

  \item set out the proper weights between SBA and Core
  \item repeat questions in mapping2, now with and without TP dish
  \item ...
\end{itemize}

\section{Results}

The results from the ngVLA presentation (in terms of Fidelity)
\begin{itemize}
  \item core:  26
  \item TP18 + core: 31
  \item TP18 + SBA + core: 45
  \item TP45 + core: 62
  \item TP100 + core: 162
\end{itemize}



\subsection{Feather vs. Joint Deconvolution}

The ``feather'' method does a very good job (by design) of fixing the total flux. There is
very little (meaningless?) variation, like for example with the Fidelity, as one varies
the dish size or other parameters to determine an optimal value for said dish size.
Other means to compare the results are the amount of positive and negative flux,
and the peak flux. This better addresses the ``missing flux'' problem so central to the
issue of missing short spacings.

A joint deconvolution approach for a mosaiced field works very differently, as it solves the
FFT in each field separately, a more honest approach. Whereas the ``feather'' method in those
cases will take

We could expect that for a single mosaiced field, feather and joint deconvolution give more similar results

It is unclear as of this writing, if the JD method will be discussed here, as then the paper becomes more
one of comparing the two methods (but in the end, isn't the JD where the better images should come from?).
It may be better to postpone this approach for the journal paper, not the white paper.



\section*{Appendix A: QAC (Quick Array Combinations)}

To run a large suite of simulations, it can be very useful to call CASA from the Unix command line,
and loop over many parameters, e.g.

\footnotesize
\begin{verbatim}
  casa --nogui -c vla1.py pixel_m=0.05 niter='[0,5000,15000]' dish=45 pdir='"exp102"' 
\end{verbatim}
\normalsize

As one of the products of the ``tp2vis'' ALMA development study (Teuben 2019) we continued the development of the
Quick Array Combination (QAC) toolkit that simplifies writing some of these complex scripts. It also allow us
to use a different combination
method (feather, tp2vis, ssc etc.) with minimal changes to the simulations scripts.

As an example, consider the {\tt simplenoise} procedure to add a given noise to a simulation.
Here is the example calling {\tt qac\_vla()} twice, in the end generating a Measurement Set with
the correct 1 mJy/beam noise:

\footnotesize
\begin{verbatim}

  rms = 0.002                                #  request 2 mJy/beam RMS noise (NA)
  ms1 = qac_vla(pdir,model, noise=-rms)      #  noise<0 triggers it to compute the rms
  sn0 = qac_noise(noise,pdir+'/noise', ms1)  #  get scaling factor from rms in ms1
  ms2 = qac_vla(pdir,model, noise=sn0)       #  MS that with correct "rms" in Jy/beam
\end{verbatim}
\normalsize

In the first Measurement Set a noise level is computed for a fixed 1 Jy noise per visiblity on a
zero model. The noise in the resulting dirty map, computed in {\tt qac\_noise()}, is then the scaling factor ({\tt sn0}
that needs to be applied to get the correct requested noise level in the second Measurement Set.











\acknowledgements We thank Alberto Bolatto, Danny Dale, Lee Mundy ...
% Keep this text on the same line as the \verb"\acknowledgements" command because it makes things a lot easier.

%\bibliography{editor}  % For BibTex

% For non-BibTex:
\begin{thebibliography}{}

\bibitem[Carilli (2017)]{memo12} Carilli, C., 2017, ngVLA Memo \#12

% resolution and sensitivity of ngvla-revB  
\bibitem[Carilli (2018)]{memo43} Carilli, C., 2018, ngVLA Memo \#43

% Koda & Teuben (2018, in prep). TP2VIS: Joint-Deconvolution of ALMA 12m, 7m and TP data - A current ALMA study  [to be delivered 31-jan-2018]

% Short Spacing Considerations for the ngVLA - ngVLA memo #14  
\bibitem[Frayer (2017)]{memo14} Frayer, A., 2017, ngVLA memo \#14


% Mason (2013), Continuum Sensitivity of GBT Receivers - GBT memo #282
\bibitem[Mason (2013)]{memo282} Mason, B. 2013, GBT memo \#282

% ngVLA short baseline array  
\bibitem[Mason et al. (2018)]{memo43} Mason, B. et al. 2018, ngVLA memo \#43

%  relative integration times 12m, 7m, TPA
\bibitem[Mason and Brogan (2017)]{memo598} Mason, B. and Brogan, C. 2017, ALMA memo \#598
  


%  impact of ACA on the wide field imaging capaboilities of ALMA

\bibitem[Pety et al. (2001)]{memo398} Pety, J. et al. 2001, ALMA memo \#398
  
  
\end{thebibliography}

\end{document}




  

  
