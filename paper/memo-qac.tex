% a longer version of this was submitted for the ngVLA project as a memo.

\documentclass[11pt,twoside]{article}
\usepackage{asp2014}
\aspSuppressVolSlug
\resetcounters

\bibliographystyle{asp2014}

\markboth{Teuben}{QAC}

\begin{document}

\title{ngVLA memo XXX: QAC: Quick Array Combinations with CASA}
\author{Peter Teuben
  \affil{Astronomy Department, University of Maryland, College Park, MD, USA}}

\paperauthor{Teuben~Peter}{teuben@astro.umd.edu}{0000-0003-1774-3436}{Astronomy Department}{University of Maryland}{College Park}{MD}{20742}{USA}

%\aindex{Teuben,~P.~J.}

\begin{abstract}

A simple python layer in CASA was developed to aid in writing scripts
for array (single dish and interferometric) combination. Although
initially developed for TP2VIS, running simulations and comparing
with other array combination methods, and adding regressions became
cumbersome, and QAC was developed. Both ALMA, ngVLA and CARMA simulati
ons are already supported, but extending to more generic array are
planned.

\end{abstract}

%\ooindex{CASA, ascl:1107.013} 

%\ssindex{instruments!interferometer}
%\ssindex{astronomy!radio!single-dish}
%\ssindex{packages!Common Astronomy Software Applications (CASA)}


\section{Introduction}

CASA (\citet{casa1} and in this volume \citet{casa2}) 
is a general purpose python interface to radio
astronomy software. It handles interferometric as well as single dish
observations, all the way from calibration and mapping to
analysis. Most ALMA and VLA data are now routinely processed with CASA
using a pipeline.  Users that use CASA can use object oriented
``tools'', or more classic python functions, called
``tasks''. One can write very complex tasks this way.

The effect of adding short spacings can be dramatic.

In a previous project, the ADMIT project had a similar challenge, but
additional more complex boundary conditions.

The development of QAC started with the TP2VIS project, to provide a
more easily programmable interface. It can be obtained from
\url{https://github.com/teuben/QAC}.

\section{Install}

The traditional way to add software to a python based system (which CASA is), would be the
python {\tt import} command, but in the current version of CASA\footnote{this may change}
the more pragmatic approach is to ``execfile'' a script, which either adds an API to the
environment and/or executes code.

Incidentally, if these are combined and only one script needs to be executed and then analyzed
outside of CASA, a very efficient way it to use could be to call casa from the command line

\footnotesize
\begin{verbatim}
    % casa --nogui -c myscript.py a=1 b='"b"' c='[1,2,3]' > myscript.log 2&>1
\end{verbatim}
\normalsize

There are a number of ways how QAC can be installed (integrated) into CASA.
Traditionally users would use
``{\tt     execfile('myscript.py')}''
from the CASA (i.e. python ) prompt. But if this is all the user needs, it can
also be replaced by a unix command line version
``{\tt     casa -c myscript.py}''
instead. The overhead of setting up casa before this script really starts work
varies a lot depending on cashing and what's in the casa init files, but can
be anywhere from 5 to 20 seconds, comparable to booting a laptop.

The installation of CASA is required to run QAC, and the user can opt either to install a version
of CASA from within QAC, or assume a version of CASA that is present on the system.

\section{Example}

A typical usage would be

\begin{verbatim}
    % casa --nogui -c 
\end{verbatim}


\section{docs}

A typical simulation script might look as follows. Explanations follow later:

\footnotesize
\begin{verbatim}
    qac_ptr(phasecenter,"test123.ptg")
    qac_vla("test123","skymodel.fits", 4096, 0.01, ptg="test123.ptg",
            phasecenter=phasecenter)
            
    qac_clean1("test123/clean1",phasecenter=phasecenter)
\end{verbatim}
\normalsize

\section{Timing and Regression}

Because QAC deal almost exclusively with image type data, the regression test is invoked automatically
with the statistics report, if a regression string (!) is given, viz.

\footnotesize
\begin{verbatim}
r = "0.0038324084555372423 0.021439742878458009 -0.048513446003198624
     0.41929447650909424 383.60327838373536"
qac_stats(test+'/clean/tpint.image')
qac_stats(test+'/clean/tpint_4.tweak.image', r)
\end{verbatim}
\normalsize

where in the first instance only the statistics are reported, the second instance will also flag any deviations.
The numbers represent the mean, std, min, max and total flux of the image.

\section{Benchmarks}

A better supported show of QAC functionality is currently in the **test/bench.py, bench0.py** and **sky1.py** rout
ines [March 2018] as those were used in the
[SD2018](\url{https://github.com/teuben/sd2018}) workshop. Please note the software in that repo is not maintained anymo
re, and updated versions can be found
within QAC.

\section{API}




\section{Future}

CASA is a development project, the next
release will have a major overhaul how python and the C++ libraries
are integrated, and this will likely have some effect how QAC is
installed, although less on its API. 

\acknowledgements Jordan Turner and Sara Negussie have been patient contributers and users.
Part of QAC was developed under the ALMA development study ``TP2VIS''  (PI: Jin Koda) and
the ``ngVLA'' array combination study (ngVLA memo 54)

\bibliography{qac}

\end{document}
